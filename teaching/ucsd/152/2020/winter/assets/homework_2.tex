     
\documentclass[addpoints]{exam}
    


\usepackage{amsfonts}
\usepackage{tikz}
\usepackage{amsmath}

\usetikzlibrary{arrows,shapes}

\sloppy



\begin{document}

  \pagestyle{headandfoot}
  \runningheadrule
  \firstpageheader{ Math 152 }{ Homework 2 }{ January 8, 2020 }
  \runningheader{ Math 152 }{ Homework 2, Page \thepage\ of \numpages}{ January 8, 2020 }

  \firstpagefooter{}{}{}
  \runningfooter{}{}{}
  \begin{flushright}
    \makebox[0.4\textwidth]{Name: \enspace\hrulefill}

    \vspace{0.2in}
    \makebox[0.4\textwidth]{Pid:\enspace\hrulefill}
  \end{flushright}

  \begin{questions}
    \question[10]
      We say that $L$ is a $B$-decision list 
			\begin{description}
			  \item[(base case)] if either $L$ is a number $y \in \mathbb{Z}$, or
			  \item[(recursion step)] $L$ is equal to $(f, v, L')$ where $f : \mathbb{Z} \to
			  \{0, 1\}$, $v \in \mathbb{Z}$, and $L$ is a $B$-decision list.
			\end{description}
			
			We can also define the value $\mathrm{val}(L, x)$ of a $B$-decision list $L$ at
			$x \in \mathbb{Z}$.
			\begin{description}
			  \item[(base case)] If $L$ is a number $y$, then $\mathrm{val}(L, x) = y$, and
			  \item[(recursion step)] if $L = (f, v, L')$, then
			    \[
			      \mathrm{val}(L, x) = 
			      \begin{cases}
			        v & \text{if } f(x) = 1 \\
			        \mathrm{val}(L', x) & \text{otherwise}
			      \end{cases}.
			    \]
			\end{description}
			
			Similarly one may define the length $\ell(L)$ of a $B$-decition list $L$.
			\begin{description}
			  \item[(base case)] If $L$ is a number $y$, then $\ell(L) = 1$, and
			  \item[(recursion step)] if $L = (f, v, L')$, then $\ell(L) = \ell(L') + 1$.
			\end{description}
			
			Assume that $\mathrm{val}(L, x) = x$ for any $x \in [1000]$ show that 
			$\ell(L) \ge 1000$.

      \begin{solution}[\stretch{1}]
      \end{solution}
      \newpage
    \question[10]
      Let $S$ be the minimal set such that $3 \in S$ and $(x + y) \in S$ for any $x, y \in S$.
			(In other words, $S$ is generated by $\{f\}$ from $3$, where $f(x, y) = x + y$.)
			Show that $S = \{3k ~:~ k \in \mathbb{N}\}$.

      \begin{solution}[\stretch{1}]
      \end{solution}
      \newpage
  \end{questions}
\end{document}