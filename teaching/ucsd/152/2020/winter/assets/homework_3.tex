     
\documentclass[addpoints]{exam}
    


\usepackage{amsfonts}
\usepackage{tikz}
\usepackage{amsmath}

\usetikzlibrary{arrows,shapes}

\sloppy



\begin{document}

  \pagestyle{headandfoot}
  \runningheadrule
  \firstpageheader{ Math 152 }{ Homework 3 }{ February 6, 2020 }
  \runningheader{ Math 152 }{ Homework 3, Page \thepage\ of \numpages}{ February 6, 2020 }

  \firstpagefooter{}{}{}
  \runningfooter{}{}{}
  \begin{flushright}
    \makebox[0.4\textwidth]{Name: \enspace\hrulefill}

    \vspace{0.2in}
    \makebox[0.4\textwidth]{Pid:\enspace\hrulefill}
  \end{flushright}

  \begin{questions}
    \question[10]
      Let $m_1, n_1, m_2, n_2 \in \mathbb{N}$, we say that $(m_1, n_1) \le (m_2, n_2)$
			iff either $m_1 < m_2$ or $m_1 = m_2$ and $n_1 < n_2$.
			
			Let $P(m, n)$ be some property of pairs of integers. Assume that we can prove
			the following statement for all $m, n \in \mathbb{N}$:
			\begin{center}
			  if $P(x, y)$ is true for all $x, y \in \mathbb{N}$ such that 
			  $(x, y) < (m, n)$, then $P(m, n)$ is true.
			\end{center}
			Show that we can prove that $P(m, n)$ is true for all $m, n \in \mathbb{N}$.

      \begin{solution}[\stretch{1}]
      \end{solution}
      \newpage
    \question[10]
      In the subtraction game where players may subtract 1, 2 or 5 chips on their
			turn, identify the N- and P-positions.

      \begin{solution}[\stretch{1}]
      \end{solution}
      \newpage
  \end{questions}
\end{document}