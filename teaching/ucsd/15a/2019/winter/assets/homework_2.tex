     
\documentclass[addpoints]{exam}
    


\usepackage{amsfonts}
\usepackage{amsmath}
\sloppy


\begin{document}

  \pagestyle{headandfoot}
  \runningheadrule
  \firstpageheader{ Math 15A }{ Homework 2 }{ January 17, 2019 }
  \runningheader{ Math 15A }{ Homework 2, Page \thepage\ of \numpages}{ January 17, 2019 }

  \firstpagefooter{}{}{}
  \runningfooter{}{}{}
  \begin{flushright}
    \makebox[0.4\textwidth]{Name: \enspace\hrulefill}

    \vspace{0.2in}
    \makebox[0.4\textwidth]{Pid:\enspace\hrulefill}
  \end{flushright}

  \begin{questions}
    \question[10]
      Let us consider four-lines geometry, it is a theory with undefined
			terms: point, line, is on, and axioms:
			\begin{enumerate}
			    \item there exist exactly four lines,
			    \item any two distinct lines have exactly one point on both of them, and
			    \item each point is on exactly two lines.
			\end{enumerate}
			
			Show that every line has exactly three points on it.

      \begin{solution}[\stretch{1}]
      \end{solution}
      \newpage
    \question[10]
      In Euclidean (standard) geometry, prove: If two lines share a common
			perpendicular, then the lines are parallel.

      \begin{solution}[\stretch{1}]
      \end{solution}
      \newpage
    \question[10]
      Show that for any positive integer $n$,
			$\sum_{i = 0}^n x^i = \frac{1 - x^{n + 1}}{1 - x}$.

      \begin{solution}[\stretch{1}]
      \end{solution}
      \newpage
  \end{questions}
\end{document}