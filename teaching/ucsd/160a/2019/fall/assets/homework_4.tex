     
\documentclass[addpoints]{exam}
    


\usepackage{amsfonts}
\usepackage{tikz}
\usepackage{amsmath}

\usetikzlibrary{arrows,shapes}

\sloppy



\begin{document}

  \pagestyle{headandfoot}
  \runningheadrule
  \firstpageheader{ Math 160A }{ Homework 4 }{ November 23, 2019 }
  \runningheader{ Math 160A }{ Homework 4, Page \thepage\ of \numpages}{ November 23, 2019 }

  \firstpagefooter{}{}{}
  \runningfooter{}{}{}
  \begin{flushright}
    \makebox[0.4\textwidth]{Name: \enspace\hrulefill}

    \vspace{0.2in}
    \makebox[0.4\textwidth]{Pid:\enspace\hrulefill}
  \end{flushright}

  \begin{questions}
    \question[10]
      Let us consider a signature $(I, <; 0)$, where $I$ is a unary relation
			intended to mean ``is interesting'', $<$ is a binary relation
			intended to mean ``is less than'', and $0$ is a constant (a function with
			zero arguments).
			
			Translate into this language the English sentences listed below.
			If the English sentence is ambiguous, you will need more than one translation.
			\begin{itemize}
			    \item Zero is less than any number.
			    \item If any number is interesting, then zero is interesting.
			    \item No number is less than zero.
			    \item Any uninteresting number with the property that all
			        smaller numbers are interesting certainly is interesting.
			    \item There is no number such that all numbers are less than it.
			    \item There is no number such that no number is less than it.
			\end{itemize}

      \begin{solution}[\stretch{1}]
      \end{solution}
      \newpage
    \question[10]
      Let us consider a signature $\mathcal{S} = (=; +, \cdot)$, where predicates and
			functions are binary.  Let $\mathfrak{M} = (\mathbb{N}; =; +, \cdot)$ be a
			structure.
			
			\begin{itemize}
			    \item Write a formula $\phi$ depending on $x$ such that for any assignment
			        $s$, $\mathfrak{M} \models \phi[s]$ iff $s(x) = 1$.
			    \item Write a formula $\phi$ depending on $x$ and $y$ such that for any
			        assignment $s$, $\mathfrak{M} \models \phi[s]$ iff $s(x) \le s(y)$.
			\end{itemize}

      \begin{solution}[\stretch{1}]
      \end{solution}
      \newpage
  \end{questions}
\end{document}