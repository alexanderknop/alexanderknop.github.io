     
\documentclass[addpoints]{exam}
    


\usepackage{amsfonts}
\usepackage{tikz}
\usepackage{amsmath}

\usetikzlibrary{arrows,shapes}

\sloppy



\begin{document}

  \pagestyle{headandfoot}
  \runningheadrule
  \firstpageheader{ Math 152 }{ Homework 1 }{ January 8, 2020 }
  \runningheader{ Math 152 }{ Homework 1, Page \thepage\ of \numpages}{ January 8, 2020 }

  \firstpagefooter{}{}{}
  \runningfooter{}{}{}
  \begin{flushright}
    \makebox[0.4\textwidth]{Name: \enspace\hrulefill}

    \vspace{0.2in}
    \makebox[0.4\textwidth]{Pid:\enspace\hrulefill}
  \end{flushright}

  \begin{questions}
    \question[10]
      Let us consider four-lines geometry, it is a theory with undefined
			terms: ``point'', ``line'', ``is on'', and axioms:
			\begin{enumerate}
			    \item there exist exactly four lines,
			    \item any two distinct lines have exactly one point on both of them, and
			    \item each point is on exactly two lines.
			\end{enumerate}
			
			Show that every line has exactly three points on it.
			(Be careful with the terms you use and axioms you use.)

      \begin{solution}[\stretch{1}]
      \end{solution}
      \newpage
    \question[10]
      In Euclidean (standard) geometry, prove: If two lines share a common
			perpendicular, then the lines are parallel.
			(You do not need to use axioms of Euclidean geometry in this exercsise, you
			can use all the standard knowledge about geometry.)

      \begin{solution}[\stretch{1}]
      \end{solution}
      \newpage
  \end{questions}
\end{document}