     
\documentclass[addpoints]{exam}
    


\usepackage{amsfonts}
\usepackage{tikz}
\usepackage{amsmath}

\usetikzlibrary{arrows,shapes}

\sloppy


\begin{document}

  \pagestyle{headandfoot}
  \runningheadrule
  \firstpageheader{ Math 160A }{ Homework 2 }{ October 26, 2019 }
  \runningheader{ Math 160A }{ Homework 2, Page \thepage\ of \numpages}{ October 26, 2019 }

  \firstpagefooter{}{}{}
  \runningfooter{}{}{}
  \begin{flushright}
    \makebox[0.4\textwidth]{Name: \enspace\hrulefill}

    \vspace{0.2in}
    \makebox[0.4\textwidth]{Pid:\enspace\hrulefill}
  \end{flushright}

  \begin{questions}
    \question
      

      \begin{parts}
        \part[10]
            Let $\phi$, $\psi$, and $\chi$ be propositional formulas on $\Omega$.
				Show that $\big(\phi \lor (\psi \land \chi)\big) \big\rvert_\rho =
				    \big((\phi \lor \psi) \land (\phi \lor \chi)\big) \big\rvert_\rho$ for any assignment
				$\rho$ to the variables $\Omega$.
            \begin{solution}[\stretch{1}]
              
            \end{solution}
        \part[10]
            Let $\psi_{1, 1}$, \dots, $\psi_{1, n}$, $\psi_{2, 1}$, \dots, $\psi_{2, m}$ be
				propositional formulas on $\Omega$.
				Let $\phi_1 = \bigwedge_{i = 1}^n \psi_{1, i}$ and $\phi_2 = \bigwedge_{j = 1}^m \psi_{2, j}$.
				
				Show that $\big(\phi_1 \lor \phi_2\big) \big\rvert_\rho =
				    \big(\bigwedge_{i = 1}^n \bigwedge_{j = 1}^m
				        (\psi_{1, i} \lor \psi_{2, j})\big) \big\rvert_\rho$ for any assignment
				$\rho$ to the variables $\Omega$.
            \begin{solution}[\stretch{1}]
              
            \end{solution}
        \part[10]
            Let $\Omega$ be a set of variables. We say that a propositional formula is
				a literal if the formula is equal to $x$ or $\lnot x$ for $x \in \Omega$.
				
				We say that a propositional formula on $\Omega$ is in conjunctive normal form
				if it is equal to $\bigwedge_{i = 1}^n \bigvee_{j = 1}^{m_i} \psi_{i, j}$,
				where $\psi_{i, j}$ is a literal.
				
				Let $\phi$ be a propositional formula on $\Omega$. Show using structural induction
				that there is a propositional formula $\psi$ on $\Omega$ in conjunctive normal
				form such that $\psi\big\rvert_\rho = \phi\big\rvert_\rho$ for any
				assignment $\rho$ to $\Omega$.
            \begin{solution}[\stretch{1}]
              
            \end{solution}
      \end{parts}
      \newpage
  \end{questions}
\end{document}