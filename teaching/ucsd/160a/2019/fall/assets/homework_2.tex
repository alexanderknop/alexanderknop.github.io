     
\documentclass[addpoints]{exam}
    


\usepackage{amsfonts}
\usepackage{tikz}
\usepackage{amsmath}

\usetikzlibrary{arrows,shapes}

\sloppy



\begin{document}

  \pagestyle{headandfoot}
  \runningheadrule
  \firstpageheader{ Math 160A }{ Homework 2 }{ October 26, 2019 }
  \runningheader{ Math 160A }{ Homework 2, Page \thepage\ of \numpages}{ October 26, 2019 }

  \firstpagefooter{}{}{}
  \runningfooter{}{}{}
  \begin{flushright}
    \makebox[0.4\textwidth]{Name: \enspace\hrulefill}

    \vspace{0.2in}
    \makebox[0.4\textwidth]{Pid:\enspace\hrulefill}
  \end{flushright}

  \begin{questions}
    \question
      

      \begin{parts}
        \part[10]
            Let $\phi$, $\psi$, and $\chi$ be propositional formulas on $\Omega$.
				Show that $\big(\phi \lor (\psi \land \chi)\big) \big\rvert_\rho =
				    \big((\phi \lor \psi) \land (\phi \lor \chi)\big) \big\rvert_\rho$ for any assignment
				$\rho$ to the variables $\Omega$.
            \begin{solution}[\stretch{1}]
              Let us fix some propositional assignment $\rho$ to $\Omega$.
Note that by the definition
\[
    \big(\phi \lor (\psi \land \chi)\big) \big\rvert_\rho =
    \phi\big\rvert_\rho \lor (\psi \land \chi) \big\rvert_\rho =
    \phi\big\rvert_\rho \lor (\psi\big\rvert_\rho \land \chi\big\rvert_\rho)
\]
and
\[
    \big((\phi \lor \psi) \land (\phi \lor \chi)\big) \big\rvert_\rho =
    (\phi \lor \psi)\big\rvert_\rho \land (\phi \lor \chi) \big\rvert_\rho =
    (\phi\big\rvert_\rho \lor \psi\big\rvert_\rho) \land
        (\phi\big\rvert_\rho \lor \chi\big\rvert_\rho).
\]
However,
$\phi\big\rvert_\rho \lor (\psi\big\rvert_\rho \land \chi\big\rvert_\rho) =
(\phi\big\rvert_\rho \lor \psi\big\rvert_\rho) \land
    (\phi\big\rvert_\rho \lor \chi\big\rvert_\rho)$ by the distributivity of
disjunction and conjunction.
            \end{solution}
        \part[10]
            Let $\psi_{1, 1}$, \dots, $\psi_{1, n}$, $\psi_{2, 1}$, \dots, $\psi_{2, m}$ be
				propositional formulas on $\Omega$.
				Let $\phi_1 = \bigwedge_{i = 1}^n \psi_{1, i}$ and $\phi_2 = \bigwedge_{j = 1}^m \psi_{2, j}$.
				
				Show that $\big(\phi_1 \lor \phi_2\big) \big\rvert_\rho =
				    \big(\bigwedge_{i = 1}^n \bigwedge_{j = 1}^m
				        (\psi_{1, i} \lor \psi_{2, j})\big) \big\rvert_\rho$ for any assignment
				$\rho$ to the variables $\Omega$.
            \begin{solution}[\stretch{1}]
              Let us again fix some propositional assignment $\rho$ to $\Omega$.

We prove the statement in two steps. In the first one we prove that
\[
    \left.\left(
        \phi_1 \lor \chi
    \right)\right\rvert_\rho =
    \left.\left(
        \bigwedge_{i = 1}^{n} \left(\psi_{1, i} \lor \chi\right)
    \right)\right\rvert_\rho
\]
using induction by $n$.

The base case for $n = 1$ is clear. Let us now prove the induction step from
$k$ to $k + 1$. Note that
\[
    \left(
        \bigwedge_{i = 1}^{k + 1} \psi_{1, i}
    \right) \lor \chi =
    \left(
        \left(
            \bigwedge_{i = 1}^{k} \psi_{1, i}
        \right) \land \psi_{1, k + 1}
    \right) \lor
        \chi.
\]
By the previous problem, this implies that
\[
    \left.\left(
        \left(
            \bigwedge_{i = 1}^{k + 1} \psi_{1, i}
        \right) \lor \chi
    \right)\right\rvert_\rho =
    \left.\left(
        \left(
            (\bigwedge_{i = 1}^{k} \psi_{1, i}) \lor \chi
        \right) \land
        \left(
            \psi_{1, k + 1} \lor \chi
        \right)
    \right)\right\rvert_\rho.
\]
The induction hypothesis, implies that
\[
    \left.\left(
        \left(
            \bigwedge_{i = 1}^{k + 1} \psi_{1, i}
        \right) \lor \chi
    \right)\right\rvert_\rho =
    \left.\left(
        \left(
            \bigwedge_{i = 1}^{k} \left(\psi_{1, i} \lor \chi\right)
        \right) \land
        \left(
            \psi_{1, k + 1} \lor \chi
        \right)
    \right)\right\rvert_\rho.
\]
Therefore, using the definition of the conjunction of several formulas,
we proved that
\[
    \left.\left(
        \left(
            \bigwedge_{i = 1}^{k + 1} \psi_{1, i}
        \right) \lor \chi
    \right)\right\rvert_\rho =
    \left.\left(
        \bigwedge_{i = 1}^{k + 1} \left(\psi_{1, i} \lor \chi\right)
    \right)\right\rvert_\rho.
\]

On the second step we prove the statement of the problem using induction by $m$.
The base case follows from the result we just proved.
Let us now prove the induction step from $k$ to $k + 1$.
Note that
\[
    \left(
        \bigwedge_{i = 1}^m \psi_{1, i}
    \right) \lor
    \left(
        \bigwedge_{i = 1}^{k + 1} \psi_{2, i}
    \right) =
    \left(
        \bigwedge_{i = 1}^m \psi_{1, i}
    \right) \lor
    \left(
        \left(
            \bigwedge_{i = 1}^{k} \psi_{2, i}
        \right) \land \psi_{2, k + 1}
    \right).
\]

By the previous problem,
\[
    \left.\left(
        \left(
            \bigwedge_{i = 1}^m \psi_{1, i}
            \right) \lor
        \left(
            \bigwedge_{i = 1}^{k + 1} \psi_{2, i}
        \right)
    \right)\right\rvert_\rho =
    \left.\left(
        \left(
            \left(
                \bigwedge_{i = 1}^m \psi_{1, i}
            \right) \lor
            \psi_{2, k + 1}
        \right) \land
        \left(
            \left(
                \bigwedge_{i = 1}^m \psi_{1, i}
            \right) \lor
            \left(
                \bigwedge_{i = 1}^{k} \psi_{2, i}
            \right)
        \right)
    \right)\right\rvert_\rho.
\]
Therefore by the previous statement,
\[
    \left.\left(
        \left(
            \bigwedge_{i = 1}^m \psi_{1, i}
            \right) \lor
        \left(
            \bigwedge_{i = 1}^{k + 1} \psi_{2, i}
        \right)
    \right)\right\rvert_\rho =
    \left.\left(
        \left(
            \bigwedge_{i = 1}^m (
                \psi_{1, i} \lor \psi_{2, k + 1}
            )
        \right) \land
        \left(
            \left(
                \bigwedge_{i = 1}^m \psi_{1, i}
            \right) \lor
            \left(
                \bigwedge_{i = 1}^{k} \psi_{2, i}
            \right)
        \right)
    \right)\right\rvert_\rho.
\]
Finally, using the induction hypothesis,
\[
    \left.\left(
        \left(
            \bigwedge_{i = 1}^m \psi_{1, i}
            \right) \lor
        \left(
            \bigwedge_{i = 1}^{k + 1} \psi_{2, i}
        \right)
    \right)\right\rvert_\rho =
    \left.\left(
        \left(
            \bigwedge_{i = 1}^m (
                \psi_{1, i} \lor \psi_{2, k + 1}
            )
        \right) \land
        \left(
            \bigwedge_{i = 1}^m \bigwedge_{i = 1}^{k} (
                \psi_{1, i} \lor \psi_{2, i}
            )            
        \right)
    \right)\right\rvert_\rho.
\]
            \end{solution}
        \part[10]
            Let $\Omega$ be a set of variables. We say that a propositional formula is
				a literal if the formula is equal to $x$ or $\lnot x$ for $x \in \Omega$.
				
				We say that a propositional formula on $\Omega$ is in conjunctive normal form
				if it is equal to $\bigwedge_{i = 1}^n \bigvee_{j = 1}^{m_i} \psi_{i, j}$,
				where $\psi_{i, j}$ is a literal.
				
				Let $\phi$ be a propositional formula on $\Omega$. Show using structural induction
				that there is a propositional formula $\psi$ on $\Omega$ in conjunctive normal
				form such that $\psi\big\rvert_\rho = \phi\big\rvert_\rho$ for any
				assignment $\rho$ to $\Omega$.
            \begin{solution}[\stretch{1}]
              Before we prove the statement of the problem, we need to show several equalities.

Let $\chi_{1, 1}$, \dots, $\chi_{1, n_1}$,
$\chi_{2, 1}$, \dots, $\chi_{2, n_2}$, $\chi_1$, \dots, $\chi_{n_1 + n_2}$ be
propositional formulas  over the variables from $\Omega$ such that
$\chi_i = \chi_{1, i}$ for $1 \le i \le n_1$ and $\chi_i = \chi_{2, i - n_1}$
for $n_1 < i \le n_1 + n_2$.
Then for any propositional assignment $\rho$ to $\Omega$,
\[
    \left.\left(
        \left(\bigwedge_{i = 1}^{n_1} \chi_{1, i}\right) \land
        \left(\bigwedge_{i = 1}^{n_2} \chi_{2, i}\right)
    \right)\right\rvert_\rho =
    \left.\left(
        \bigwedge_{i = 1}^{n_1 + n_2} \chi_{i}
    \right)\right\rvert_\rho.
\]
We can prove the statement using induction by $n_2$. The base case for $n_2 = 1$
follows from the definition of the long conjunction.
Let us prove the induction step from $k$ to $k + 1$.
By the definition of the long conjunction,
\[
    \left(\bigwedge_{i = 1}^{n_1} \chi_{1, i}\right) \land
    \left(\bigwedge_{i = 1}^{k + 1} \chi_{2, i}\right)
    =
    \left(\bigwedge_{i = 1}^{n_1} \chi_{1, i}\right) \land
    \left(
        \left(\bigwedge_{i = 1}^k \chi_{2, i}\right) \land
        \chi_{2, k + 1}
    \right)
\]
Note that we proved in class the following
\begin{multline*}
    \left.\left(
        \left(\bigwedge_{i = 1}^{n_1} \chi_{1, i}\right) \land
        \left(\bigwedge_{i = 1}^{k + 1} \chi_{2, i}\right)
    \right)\right\rvert_\rho
    =
    \left.\left(
        \left(\bigwedge_{i = 1}^{n_1} \chi_{1, i}\right) \land
        \left(
            \left(\bigwedge_{i = 1}^k \chi_{2, i}\right) \land
            \chi_{2, k + 1}
        \right)
    \right)\right\rvert_\rho
    = \\
    \left.\left(
        \left(
            \left(\bigwedge_{i = 1}^{n_1} \chi_{1, i}\right) \land
            \left(\bigwedge_{i = 1}^k \chi_{2, i}\right)
        \right) \land
        \chi_{2, k + 1}
    \right)\right\rvert_\rho.
\end{multline*}
By the induction hypothesis,
\[
    \left.\left(
        \left(\bigwedge_{i = 1}^{n_1} \chi_{1, i}\right) \land
        \left(\bigwedge_{i = 1}^{k + 1} \chi_{2, i}\right)
    \right)\right\rvert_\rho
    =
    \left.\left(
        \left(
            \bigwedge_{i = 1}^{n_1 + k} \chi_{i}
        \right) \land
        \chi_{2, k + 1}
    \right)\right\rvert_\rho.
\]
Which implies the statement by the definition of the long conjunction.

Consider $e_{n, m} : \{0, \dots, m^n - 1\} \to \{0, \dots, m - 1\}^n$ be a bijection
such that
\begin{itemize}
  \item $e_{n, m}(i + m \cdot r, 0) = i$ and
  \item $e_{n, m}(i + m \cdot r, j) = e_{n - 1, m}(r, j - 1)$,
\end{itemize}
for any $0 \le i < m$, $0 \le r < m^{n - 1}$, and $0 \le j < n$.
We also show that
$\bigvee_{j = 0}^{n - 1} \bigwedge_{i = 0}^{m - 1} \chi_{j, i} =
\bigwedge_{q = 0}^{m^n - 1} \bigvee_{j = 0}^{n - 1} \chi_{j, e_{n, m}(q, j)}$.
We prove the statement using induction by $n$. The case of $n = 1$ is clear.
We prove now the induction step from $k$ to $k + 1$.
Note that
\[
  \bigvee_{j = 0}^k \bigwedge_{i = 0}^{m - 1} \chi_{j, i} =
  \left(\bigvee_{j = 0}^{k - 1} \bigwedge_{i = 0}^{m - 1} \chi_{j, i}\right)
  \lor \bigwedge_{i = 0}^{m - 1} \chi_{k, i}.
\]
By the induction hypothesis,
\[
  \left.\left(
    \bigvee_{j = 0}^k \bigwedge_{i = 0}^{m - 1} \chi_{j, i}
  \right)\right\rvert_\rho =
  \left.\left(
    \left(
      \bigwedge_{q = 0}^{m^k - 1} \bigvee_{j = 0}^{k - 1}
        \chi_{j, e_{k, m}(q, j)}
    \right) \lor
    \bigwedge_{i = 0}^{m - 1} \chi_{k, i}
  \right)\right\rvert_\rho.
\]
Therefore,
\[
  \left.\left(
    \bigvee_{j = 0}^k \bigwedge_{i = 0}^{m - 1} \chi_{j, i}
  \right)\right\rvert_\rho =
  \left.\left(
      \bigwedge_{q = 0}^{m^k - 1}
        \left(
          \left(
            \bigvee_{j = 0}^{k - 1}
              \chi_{j, e_{k, m}(q, j)}
          \right) \lor
          \bigwedge_{i = 0}^{m - 1} \chi_{k, i}
        \right)
  \right)\right\rvert_\rho.
\]
So
\[
  \left.\left(
    \bigvee_{j = 0}^k \bigwedge_{i = 0}^{m - 1} \chi_{j, i}
  \right)\right\rvert_\rho =
  \left.\left(
      \bigwedge_{q = 0}^{m^k - 1}
      \bigwedge_{i = 0}^{m - 1}
        \left(
          \left(
            \bigvee_{j = 0}^{k - 1}
              \chi_{j, e_{k, m}(q, j)}
          \right)
          \lor
          \chi_{k, i}
        \right)
  \right)\right\rvert_\rho.
\]
As a result,
\[
  \left.\left(
    \bigvee_{j = 0}^k \bigwedge_{i = 0}^{m - 1} \chi_{j, i}
  \right)\right\rvert_\rho =
  \left.\left(
      \bigwedge_{q = 0}^{m^{k + 1} - 1}
        \left(
          \bigvee_{j = 0}^k
            \chi_{j, e_{k + 1, m}(q, j)}
        \right)
  \right)\right\rvert_\rho.
\]

Finally, we are ready to prove the statement of the problem.
Let us consider the following cases.
\begin{itemize}
    \item The first case is when $\phi = x_i$. In this case we can choose
        $n = 1$, $m_1 = 1$, and $\psi_{1, 1} = x_i$.
    \item The second case is when $\phi = \phi_1 \land \phi_2$.
        Note that by the induction hypothesis,
        $\phi_1 = \bigwedge_{j = 1}^{n_1} \bigvee_{k = 1}^{m_{1, i}} \psi_{1, i, j}$
        and
        $\phi_2 = \bigwedge_{j = 1}^{n_2} \bigvee_{k = 1}^{m_{2, i}} \psi_{2, i, j}$.
        Using the statement of the previous problem,
        $\phi = \bigwedge_{j = 1}^{n_1 + n_2} \bigvee_{k = 1}^{m_i} \psi_{i, j}$.
    \item The third case is when $\phi = \phi_1 \lor \phi_2$.
        Note that by the induction hypothesis,
        $\phi_1 = \bigwedge_{i = 1}^{n_1} \bigvee_{j = 1}^{m_{1, i}} \psi_{1, i, j}$
        and
        $\phi_2 = \bigwedge_{i = 1}^{n_2} \bigvee_{j = 1}^{m_{2, i}} \psi_{2, i, j}$.
        Using the just proved statement we may conclude that
        $\phi =
          \bigwedge_{i = 1}^{n_1 + n_2} \bigvee_{j = 1}^{m_i} \psi_{i, j}$,
        where $m_i = m_{1, i}$ for $1 \le i \le n_1$ and $m_{2, i - n_1}$ for
        $n_1 < i \le n_1 + n_2$, $\psi_{i, j} = \psi_{1, i, j}$ for
        $1 \le i \le n_1$ and $\psi_{i, j} = \psi_{2, i - n_1, j}$ for
        $n_1 < i \le n_1 + n_2$.
    \item Finally, the last case is when $\phi = \lnot \phi'$.
        Note that by the induction hypothesis,
        $\phi^\prime =
          \bigwedge_{j = 1}^{n^\prime}
            \bigvee_{k = 1}^{m^\prime_i} \psi^\prime_{i, j}$.

        Hence, $\phi\rvert_\rho =
          \bigvee_{j = 1}^{n^\prime}
            \bigwedge_{k = 1}^{m^\prime_i} \lnot \psi^\prime_{i, j}$.
        And using the second proven observation in this problem we can present
        a CNF representation of $\phi$. 
\end{itemize}
            \end{solution}
      \end{parts}
      \newpage
  \end{questions}
\end{document}