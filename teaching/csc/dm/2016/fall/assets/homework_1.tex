
\documentclass[12pt,fleqn,a4paper]{article}

\usepackage[russian]{babel}
\usepackage[utf8]{inputenc}
\usepackage{amsmath}
\usepackage{amsfonts}
\usepackage{enumitem}
\usepackage{ntheorem}
\usepackage{tikz}
\usepackage{verbatim}

\usetikzlibrary{arrows,shapes}

\tikzstyle{vertex}=[circle,fill=black,minimum size=3pt,inner sep=0pt]
\tikzstyle{edge} = [draw,thick,-]

\newtheorem{definition}{Определение}
\newtheorem*{solution}{Решение}

\newenvironment{task}[2] {
	\noindent\fbox{\bf {#1} {#2}.}
}{
}

\newcommand{\mytitle}[2] {
  \begin{center}
      \bf {#1} {#2}.
  \end{center}
}

\let\origenumerate\enumerate
\let\origendenumerate\endenumerate
\renewenvironment{enumerate}{\origenumerate[topsep = 0pt, noitemsep]}{\origendenumerate}

\begin{document}
	\mytitle{Домашнее задание 1.}{Биномиальные коэффициенты}
	\begin{task}{DM}{13}
		($1$ балл) Сколько существует шестизначных чисел, сумма цифр которых не превосходит 47?
	\end{task}
	\begin{solution}
	\end{solution}

	\begin{task}{DM}{14}
		($1$ балл) Сколько различных слов можно получить, переставляя буквы слова ``метаматематика''?
	\end{task}
	\begin{solution}
	\end{solution}

	\begin{task}{DM}{15}
		($1$ балл) Докажите \textit{тождество Вандермонда}:
		        $$\binom{n+m}{k} = \sum_{i=0}^k \binom{n}{i} \cdot \binom{m}{k-i}.$$
	\end{task}
	\begin{solution}
	\end{solution}

	\begin{task}{DM}{16}
		($1$ балл) С помощью \textit{формулы суммирования по верхнему индексу} 
		        $\sum_{m=0}^n \binom{m}{k} = \binom{n+1}{k+1}$ выразите значение следующей суммы через полином от $n$:
		        $$\sum_{i=0}^n i^3.$$
	\end{task}
	\begin{solution}
	\end{solution}

	\begin{task}{DM}{17}
		($1$ балл) Для натурального $n$, назовем $n$-разбиением числа $k$ назовём упорядоченный набор неотрицательных целых чисел $a_i$, $1 \leq i \leq n$, 
		        для которого верно, что $\sum_{i=1}^n a_i = k$. Например, $(3,0,1)$ и $(0,3,1)$ — два различных $3$-разбиения числа $4$.
		        Подсчитайте количество $n$-разбиений числа $k$, удовлетворяющих ограничениям 
		        $$a_i\geq s_i,\quad i=1,\ldots,n;\qquad \qquad s_1+s_2+\ldots+s_n=:s\leq k.$$
	\end{task}
	\begin{solution}
	\end{solution}

	\begin{task}{DM}{18}
		($1$ балл) Сколько можно построить различных прямоугольных параллелепипедов, у которых длина каждого ребра является целым числом от $1$ до $10$? 
		        Сколько можно построить треугольных пирамид, у которых все углы при одной из вершин прямые и длина каждого из рёбер при этой вершине является
		        целым числом от $1$ до $10$? Многогранники считаются различными, если их нельзя совместить с помощью параллельного переноса или поворота.
	\end{task}
	\begin{solution}
	\end{solution}

	\begin{task}{DM}{19}
		($2$ балла) Сколькими способами можно выбрать два подмножества, $A$ и $B$, $n$-элементного множества так, чтобы их пересечение было не пусто?
	\end{task}
	\begin{solution}
	\end{solution}

	\begin{task}{DM}{20}
		($2$ балла) Пусть $\widehat{S}(n,k)$ — число {\em сюрьективных отображений}, то есть число функций $f$ из $n$-элементного множества $X$ в 
		        $k$-элементное множество $Y$, таких что $\forall y \in Y \quad \exists x \in X : f(x)=y.$ Найдите явные формулы для $\widehat{S}(n,3)$ и $\widehat{S}(n,n-2)$.
	\end{task}
	\begin{solution}
	\end{solution}

	\begin{task}{DM}{21}
		($2$ балла) Докажите комбинаторно следующую формулу:
		        $$\widehat{S}(n,k)=\sum_{i=1}^n \widehat{S}(n-i,k-1) \cdot k^{i}.$$
	\end{task}
	\begin{solution}
	\end{solution}

	\begin{task}{DM}{22}
		($2$ балла) Докажите комбинаторно следующую формулу:
		        $$\widehat{S}(n,k)=k \cdot \widehat{S}(n-1,k) + k \cdot \widehat{S}(n-1,k-1).$$
		        Эта формула вполне подходит для того, чтобы вычислять значения $\widehat{S}(n,k)$ рекурсивно. 
		        Но чтобы вычисление не шло вечно, для каких-то значений аргументов нужно сразу знать ответ и не применять рекуррентную формулу. 
		        Определите начальные условия: чему равно $\widehat{S}(n,0)$, $\widehat{S}(n,n)$ и, в частности, $\widehat{S}(0,0)$?
	\end{task}
	\begin{solution}
	\end{solution}

\end{document}