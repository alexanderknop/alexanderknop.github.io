
\documentclass[12pt,fleqn,a4paper]{article}

\usepackage[russian]{babel}
\usepackage[utf8]{inputenc}
\usepackage{amsmath}
\usepackage{amsfonts}
\usepackage{enumitem}
\usepackage{ntheorem}
\usepackage{tikz}
\usepackage{verbatim}

\usetikzlibrary{arrows,shapes}

\tikzstyle{vertex}=[circle,fill=black,minimum size=3pt,inner sep=0pt]
\tikzstyle{edge} = [draw,thick,-]

\newtheorem{definition}{Определение}
\newtheorem*{solution}{Решение}

\newenvironment{task}[2] {
	\noindent\fbox{\bf {#1} {#2}.}
}{
}

\newcommand{\mytitle}[2] {
  \begin{center}
      \bf {#1} {#2}.
  \end{center}
}

\let\origenumerate\enumerate
\let\origendenumerate\endenumerate
\renewenvironment{enumerate}{\origenumerate[topsep = 0pt, noitemsep]}{\origendenumerate}

\begin{document}
	\mytitle{Домашнее задание 6.}{Теория графов}
	\begin{task}{DM}{61}
		Докажите или опровергните следующее утверждение: объединение двух различных маршрутов, соединяющих две вершины, содержит простой цикл.
	\end{task}
	\begin{solution}
	\end{solution}

	\begin{task}{DM}{62}
		Рассмотрим квадратную сетку, состоящую из $5\cdot 5=25$ вершин, соединенных между собой сорока ребрами. Можно ли покрыть эту сетку пятью ломаными длины $8$? А восемью ломаными длины $5$?
	\end{task}
	\begin{solution}
	\end{solution}

	\begin{task}{DM}{63}
		Рассмотрим простой регулярный граф $G$, степень каждой вершины которого равна четырем. Докажите, что ребра этого графа всегда можно покрасить в два цвета (красный и синий) так,
		чтобы любая вершина была инцидентна ровно двум синим и ровно двум красным ребрам.
	\end{task}
	\begin{solution}
	\end{solution}

	\begin{task}{DM}{64}
		Докажите, что в графе, изображенном на рисунке, не существует гамильтонова цикла
		(Гамильтонов цикл - это цикл проходящий по всем вершинам ровно по одному разу).
		
		\begin{figure}[h]
			\centering
			\begin{tikzpicture}[scale=0.5, auto,swap]
				% Draw a 7,11 network
				% First we draw the vertices
				\foreach \pos/\name in {{(0,0)/1}, {(3,0)/2}, {(5,0)/3},
		        	                    {(7,0)/4}, {(10,0)/5}, {(4,2)/6}, {(6,2)/7}, 
						    {(5,4)/8}, {(4,-2)/9}, {(6,-2)/10}, {(5,-4)/11}}
					\node[vertex] (\name) at \pos {};
				\foreach \source/ \dest in {1/2, 4/5, 2/6, 2/9, 3/6, 3/9, 3/7, 
						    3/10, 4/7, 4/10, 8/6, 8/7, 11/9, 11/10, 1/8, 1/11, 5/8, 5/11}
					\path[edge] (\source) -- (\dest);
			\end{tikzpicture}
		\end{figure}
	\end{task}
	\begin{solution}
	\end{solution}

\end{document}