
\documentclass[12pt,fleqn,a4paper]{article}

\usepackage[russian]{babel}
\usepackage[utf8]{inputenc}
\usepackage{amsmath}
\usepackage{amsfonts}
\usepackage{enumitem}
\usepackage{ntheorem}
\usepackage{tikz}
\usepackage{verbatim}

\usetikzlibrary{arrows,shapes}

\tikzstyle{vertex}=[circle,fill=black,minimum size=3pt,inner sep=0pt]
\tikzstyle{edge} = [draw,thick,-]

\newtheorem{definition}{Определение}
\newtheorem*{solution}{Решение}

\newenvironment{task}[2] {
	\noindent\fbox{\bf {#1} {#2}.}
}{
}

\newcommand{\mytitle}[2] {
  \begin{center}
      \bf {#1} {#2}.
  \end{center}
}

\let\origenumerate\enumerate
\let\origendenumerate\endenumerate
\renewenvironment{enumerate}{\origenumerate[topsep = 0pt, noitemsep]}{\origendenumerate}

\begin{document}
	\mytitle{Домашнее задание 2.}{Формула включений исключений и формула обращения Мебиуса}
	\begin{task}{DM}{25}
		(1 балл) В школе три спортивных команды. Для любых двух учеников найдётся команда, в которой они состоят оба. Докажите, что найдётся команда, 
		в которой состоят по меньшей мере $2/3$ учеников.
	\end{task}
	\begin{solution}
	\end{solution}

	\begin{task}{DM}{26}
		(1 балл) Сколькими различными способами можно переставить цифры $\{1,1,2,2,3,4,5\}$ так, чтобы никакие две одинаковые цифры не стояли рядом?
	\end{task}
	\begin{solution}
	\end{solution}

	\begin{task}{DM}{27}
		(1 балл) Через $[n]$ будем обозначать множество натуральных чисел от единицы до $n$. Функция Эйлера $\varphi(n)$ определяется как количество 
		чисел из $[n]$, взаимно простых с $n$. Найдите формулу для $\varphi(p^k)$, если $p$ — простое.
	\end{task}
	\begin{solution}
	\end{solution}

	\begin{task}{DM}{28}
		(1 балл) Найдите $\varphi(210)$.
	\end{task}
	\begin{solution}
	\end{solution}

	\begin{task}{DM}{29}
		(1 балл) Докажите или опровергните, что для любых натуральных $n,m$ верна формула: $$\varphi(m \cdot n)=\varphi(m)\cdot \varphi(n).$$
	\end{task}
	\begin{solution}
	\end{solution}

\end{document}