
\documentclass[12pt,fleqn,a4paper]{article}

\usepackage[russian]{babel}
\usepackage[utf8]{inputenc}
\usepackage{amsmath}
\usepackage{amsfonts}
\usepackage{enumitem}
\usepackage{ntheorem}
\usepackage{tikz}
\usepackage{verbatim}

\usetikzlibrary{arrows,shapes}

\tikzstyle{vertex}=[circle,fill=black,minimum size=3pt,inner sep=0pt]
\tikzstyle{edge} = [draw,thick,-]

\newtheorem{definition}{Определение}
\newtheorem*{solution}{Решение}

\newenvironment{task}[2] {
	\noindent\fbox{\bf {#1} {#2}.}
}{
}

\newcommand{\mytitle}[2] {
  \begin{center}
      \bf {#1} {#2}.
  \end{center}
}

\let\origenumerate\enumerate
\let\origendenumerate\endenumerate
\renewenvironment{enumerate}{\origenumerate[topsep = 0pt, noitemsep]}{\origendenumerate}

\begin{document}
	\mytitle{Домашнее задание 3.}{Частично упорядоченные множества}
	\begin{task}{DM}{33}
		(1 балл) В ящике $10$ белых и $20$ черных носков. Сколько минимум нужно вынуть носков, чтобы гарантировать, что вам удастся вынуть хотябы два 
		одного цвета.
	\end{task}
	\begin{solution}
	\end{solution}

	\begin{task}{DM}{34}
		(1 балл) Найдите такое минимальное $k$, что если мы выберем $k$ различных чисел из чисел от $1$ до $20$, то обязательно найдется пара 
		дающая в сумме $21$.
	\end{task}
	\begin{solution}
	\end{solution}

	\begin{task}{DM}{35}
		(1 балл) Пусть $\{A_i\}, i \in [k]$ — набор из $k$ подмножеств множества $[n]$. Известно, что пересечение любых двух подмножеств 
		из этого набора непусто. Докажите, что $k \leq  2^{n-1}$. Приведите пример, на котором в этом неравенстве достигается равенство.
	\end{task}
	\begin{solution}
	\end{solution}

	\begin{task}{DM}{36}
		(1 балл) Даны несколько различных натуральных чисел. Докажите, что если среди
		любых $n$ из них можно выбрать два так, что одно делится на другое, то все 
		числа можно покрасить в $n - 1$ цвет так, чтобы из любых двух чисел одного
		цвета одно делилось на другое.
	\end{task}
	\begin{solution}
	\end{solution}

	\begin{task}{DM}{37}
		(1 балл) Докажите, что любая последовательность из $n^2 + 1$ различных целых чисел содержит либо убывающую, либо возрастающую подпоследовательность 
		из не менее чем $n + 1$ числа.
	\end{task}
	\begin{solution}
	\end{solution}

	\begin{task}{DM}{38}
		(2 балла) Пусть на прямой задана произвольная система отрезков. Обозначим
		через $M$ наименьшее количество точек на прямой таких, что каждый из
		отрезков системы содержит одну из этих точек; через $m$ — наибольшее
		количество попарно непересекающихся отрезков, которые можно выбрать из
		данной системы. Докажите, что $M = m$.
	\end{task}
	\begin{solution}
	\end{solution}

	\begin{task}{DM}{39}
		(2 балла) Пусть числом Белла $B(n)$ называется число разбиений чисел от $1$ до $n$ на неупорядоченные блоки (по определению $B(0) = 1$).
		
		Доказать, что число разбиений $n$-элементного множества, при котором ни в одном блоке не содержится пара последовательно идущих чисел,
		равно числу Белла $B(n - 1)$.
	\end{task}
	\begin{solution}
	\end{solution}

\end{document}