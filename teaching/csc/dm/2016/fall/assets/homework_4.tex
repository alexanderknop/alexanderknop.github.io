
\documentclass[12pt,fleqn,a4paper]{article}

\usepackage[russian]{babel}
\usepackage[utf8]{inputenc}
\usepackage{amsmath}
\usepackage{amsfonts}
\usepackage{enumitem}
\usepackage{ntheorem}
\usepackage{tikz}
\usepackage{verbatim}

\usetikzlibrary{arrows,shapes}

\tikzstyle{vertex}=[circle,fill=black,minimum size=3pt,inner sep=0pt]
\tikzstyle{edge} = [draw,thick,-]

\newtheorem{definition}{Определение}
\newtheorem*{solution}{Решение}

\newenvironment{task}[2] {
	\noindent\fbox{\bf {#1} {#2}.}
}{
}

\newcommand{\mytitle}[2] {
  \begin{center}
      \bf {#1} {#2}.
  \end{center}
}

\let\origenumerate\enumerate
\let\origendenumerate\endenumerate
\renewenvironment{enumerate}{\origenumerate[topsep = 0pt, noitemsep]}{\origendenumerate}

\begin{document}
	\mytitle{Домашнее задание 4.}{Группа перестановок и разбиения числа в сумму слагаемых}
	\begin{task}{DM}{46}
		(1 балл) Число Стирлинга 1-го рода без знака $c(n,k)$ — это количество перестановок из $S_n$, имеющих в точности $k$ циклов. 
		Доказать равенство: $$c(n,k)=c(n-1,k-1)+(n-1)\cdot c(n-1,k).$$
	\end{task}
	\begin{solution}
	\end{solution}

	\begin{task}{DM}{47}
		(1 балл) Найдите рекуррентную формулу для числа перестановок, куб которых — тождественная перестановка.
	\end{task}
	\begin{solution}
	\end{solution}

	\begin{task}{DM}{48}
		(2 балла) Какова вероятность того, что $k$ заданных элементов принадлежат одному циклу в случайной перестановке на $n$ элементах?
	\end{task}
	\begin{solution}
	\end{solution}

	\begin{task}{DM}{49}
		(2 балла) Назовём матрицами перестановок матрицы, в которых на каждой строке и в каждом столбце стоит ровно по одной единице, причём все элементы кроме этих 
		единиц — нули. Какое минимальное количество матриц перестановок нужно зафиксировать, чтобы перемножением этих матриц (взятых в любом количестве) 
		можно было получить любую другую матрицу перестановки?
	\end{task}
	\begin{solution}
	\end{solution}

	\begin{task}{DM}{50}
		(1 балл) Докажите, что разбиений числа $n$, в которых все слагаемые не превосходят $k$, столько же, сколько разбиений $n$ на не более $k$ не нулевых слагаемых.
	\end{task}
	\begin{solution}
	\end{solution}

\end{document}