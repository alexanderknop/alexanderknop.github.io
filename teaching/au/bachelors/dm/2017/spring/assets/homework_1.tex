
\documentclass[12pt,fleqn,a4paper]{article}

\usepackage[russian]{babel}
\usepackage[utf8]{inputenc}
\usepackage{amsmath}
\usepackage{amsfonts}
\usepackage{enumitem}
\usepackage{ntheorem}
\usepackage{tikz}
\usepackage{verbatim}

\sloppy

\usetikzlibrary{arrows,shapes}

\tikzstyle{vertex}=[circle,fill=black,minimum size=3pt,inner sep=0pt]
\tikzstyle{edge} = [draw,thick,-]

\newtheorem{definition}{Определение}
\newtheorem*{solution}{Решение}

\newenvironment{task}[2] {
	\noindent\fbox{\bf {#1} {#2}.}
}{
}

\newcommand{\mytitle}[2] {
  \begin{center}
      \bf {#1} {#2}.
  \end{center}
}

\let\origenumerate\enumerate
\let\origendenumerate\endenumerate
\renewenvironment{enumerate}{\origenumerate[topsep = 0pt, noitemsep]}{\origendenumerate}

\begin{document}
	\mytitle{Домашняя работа 1.}{Принцип Дирихле}
	\begin{task}{COMB}{18}
		(1 балл) Внутри равностороннего треугольника со стороной в один сантиметр расположено пять точек. Доказать, что расстояние между хотя бы двумя из них меньше $0.5$ сантиметров
	\end{task}
	\begin{solution}
	\end{solution}

	\begin{task}{COMB}{19}
		(1 балл) Узлы бесконечной клетчатой бумаги покрашены в два цвета. 
		Доказать, что существуют две горизонтальные и две вертикальные прямые, на пересечениях которых лежат точки, покрашенные в один и тот же цвет.
	\end{task}
	\begin{solution}
	\end{solution}

	\begin{task}{COMB}{20}
		(1 балл) На плоскости нарисовано $n$ попарно не параллельных прямых. Доказать, угол между по крайней мере двумя из этих прямых меньше или равен величине $\pi/n$.
	\end{task}
	\begin{solution}
	\end{solution}

	\begin{task}{COMB}{21}
		(1 балл) Сколько существует целых чисел между $0$ и $999$, содержащих хотя бы одну цифру $7$?
	\end{task}
	\begin{solution}
	\end{solution}

	\begin{task}{COMB}{22}
		(1 балл) Доказать следующую формулу:
			$$|A \cap B \cap C| = |A| + |B| + |C| - |A \cup B| - |A \cup C| - |B \cup C| + |A \cup B \cup C|.$$
	\end{task}
	\begin{solution}
	\end{solution}

	\begin{task}{COMB}{23}
		(1 балл) Доказать комбинаторно так называемую формулу суммирования по диагонали:
		$$\sum\limits_{k = 0}^n \binom{m + k}{k} = \binom{m + m + 1}{n}.$$
	\end{task}
	\begin{solution}
	\end{solution}

	\begin{task}{COMB}{24}
		(3 балла)
		Используя формулу суммирования по верхнему индексу вычислить:
		\begin{enumerate}
			\item[(а)] $\sum\limits_{i = 0}^n i$,
			\item[(б)] $\sum\limits_{i = 0}^n i^2$,
			\item[(в)] $\sum\limits_{i = 0}^n i^3$.
		\end{enumerate}
	\end{task}
	\begin{solution}
	\end{solution}

\end{document}