
\documentclass[12pt,fleqn,a4paper]{article}

\usepackage[russian]{babel}
\usepackage[utf8]{inputenc}
\usepackage{amsmath}
\usepackage{amsfonts}
\usepackage{enumitem}
\usepackage{ntheorem}
\usepackage{tikz}
\usepackage{verbatim}

\sloppy

\usetikzlibrary{arrows,shapes}

\tikzstyle{vertex}=[circle,fill=black,minimum size=3pt,inner sep=0pt]
\tikzstyle{edge} = [draw,thick,-]

\newtheorem{definition}{Определение}
\newtheorem*{solution}{Решение}

\newenvironment{task}[2] {
	\noindent\fbox{\bf {#1} {#2}.}
}{
}

\newcommand{\mytitle}[2] {
  \begin{center}
      \bf {#1} {#2}.
  \end{center}
}

\let\origenumerate\enumerate
\let\origendenumerate\endenumerate
\renewenvironment{enumerate}{\origenumerate[topsep = 0pt, noitemsep]}{\origendenumerate}

\begin{document}
	\mytitle{Домашняя работа 2.}{Биномиальные коэффициенты}
	\begin{task}{COMB}{31}
		($1$ балл) Сколько существует шестизначных чисел, сумма цифр которых не превосходит 47?
	\end{task}
	\begin{solution}
	\end{solution}

	\begin{task}{COMB}{32}
		($1$ балл) Для натурального $n$, назовем $n$-разбиением числа $k$ назовем упорядоченный набор неотрицательных целых чисел $a_i$, $1 \leq i \leq n$, 
		для которого верно, что $\sum_{i=1}^n a_i = k$. Например, $(3,0,1)$ и $(0,3,1)$ — два различных $3$-разбиения числа $4$.
		Подсчитайте количество $n$-разбиений числа $k$, удовлетворяющих ограничениям 
		$$a_i\geq s_i,\quad i=1,\ldots,n;\qquad \qquad s_1+s_2+\ldots+s_n=:s\leq k.$$
	\end{task}
	\begin{solution}
	\end{solution}

	\begin{task}{COMB}{33}
		($2$ балла) Пусть $\widehat{S}(n,k)$ — число {\em сюрьективных отображений}, то есть число функций $f$ из $n$-элементного множества $X$ в 
		        $k$-элементное множество $Y$, таких что $\forall y \in Y \quad \exists x \in X : f(x)=y.$ Найдите явные формулы для $\widehat{S}(n,3)$ и $\widehat{S}(n,n-2)$.
	\end{task}
	\begin{solution}
	\end{solution}

	\begin{task}{COMB}{34}
		($2$ балла) Докажите комбинаторно следующую формулу:
		        $$\widehat{S}(n,k)=k \cdot \widehat{S}(n-1,k) + k \cdot \widehat{S}(n-1,k-1).$$
		        Эта формула вполне подходит для того, чтобы вычислять значения $\widehat{S}(n,k)$ рекурсивно. 
		        Но чтобы вычисление не шло вечно, для каких-то значений аргументов нужно сразу знать ответ и не применять рекуррентную формулу. 
		        Определите начальные условия: чему равно $\widehat{S}(n,0)$, $\widehat{S}(n,n)$ и, в частности, $\widehat{S}(0,0)$?
	\end{task}
	\begin{solution}
	\end{solution}

	\begin{task}{COMB}{35}
		(2 балла) Пусть числом Белла $B(n)$ называется число разбиений чисел от $1$ до $n$ на неупорядоченные блоки (по определению $B(0) = 1$).
		
		Доказать, что число разбиений $n$-элементного множества, при котором ни в одном блоке не содержится пара последовательно идущих чисел,
		равно числу Белла $B(n - 1)$.
	\end{task}
	\begin{solution}
	\end{solution}

\end{document}