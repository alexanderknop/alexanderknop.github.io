
\documentclass[12pt,fleqn,a4paper]{article}

\usepackage[russian]{babel}
\usepackage[utf8]{inputenc}
\usepackage{amsmath}
\usepackage{amsfonts}
\usepackage{enumitem}
\usepackage{ntheorem}
\usepackage{tikz}
\usepackage{verbatim}

\sloppy

\usetikzlibrary{arrows,shapes}

\tikzstyle{vertex}=[circle,fill=black,minimum size=3pt,inner sep=0pt]
\tikzstyle{edge} = [draw,thick,-]

\newtheorem{definition}{Определение}
\newtheorem*{solution}{Решение}

\newenvironment{task}[2] {
	\noindent\fbox{\bf {#1} {#2}.}
}{
}

\newcommand{\mytitle}[2] {
  \begin{center}
      \bf {#1} {#2}.
  \end{center}
}

\let\origenumerate\enumerate
\let\origendenumerate\endenumerate
\renewenvironment{enumerate}{\origenumerate[topsep = 0pt, noitemsep]}{\origendenumerate}

\begin{document}
	\mytitle{Домашняя работа 3.}{Биномиальные коэффициенты}
	\begin{task}{COMB}{41}
		(1 балл) Если $F(n)$ --- это число разбиений $n$-элементного множества без блоков единичной длины, то $B(n) = F(n) + F(n + 1)$.
	\end{task}
	\begin{solution}
	\end{solution}

	\begin{task}{COMB}{42}
		(3 балла) Найдите явные формулы для последовательностей, удовлетворяющих следующим рекуррентным соотношениям:
		\begin{enumerate}
			\item[(а)] $a_{n + 2} = 5 a_{n +  1} - 6 a_n$, $a_0 = 2$ и $a_1 = 6$;
			\item[(б)] $a_{n + 2} = -2 a_{n + 1} - a_n$, $a_0 = 2$ и $a_1 = 6$;
			\item[(в)] $a_{n + 2} = 2 \sqrt{2} a_{n + 1} - 4 a_n$, $a_0 = 1$ и $a_1 = 2$.
		\end{enumerate}
	\end{task}
	\begin{solution}
	\end{solution}

	\begin{task}{COMB}{43}
		(1.5 балла) Найдите явную формулу для последовательности, удовлетворяющей неоднородному рекуррентному соотношению $a_{n + 2} = 5 a_{n + 1} - 4 a_n + 3 \cdot 2^n$.
	\end{task}
	\begin{solution}
	\end{solution}

	\begin{task}{COMB}{44}
		(2 балла) Найдите явную формулу для последовательности удовлетворяющей неоднородному рекуррентному соотношению $a_{n + 2} = 3 a_{n + 1} - 2 a_n + 3 \sin(n \pi / 2)$.
	\end{task}
	\begin{solution}
	\end{solution}

	\begin{task}{COMB}{45}
		(1 балл) Рассмотрим следующее блуждание по плоскости: мы можем делать шаг вверх, вправо и влево на один, при этом запрещено сделать шаг вправо, а потом шаг влево (и наоборот).
		Сколько способов сделать $n$ шагов.
	\end{task}
	\begin{solution}
	\end{solution}

\end{document}