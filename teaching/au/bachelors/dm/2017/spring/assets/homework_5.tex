
\documentclass[12pt,fleqn,a4paper]{article}

\usepackage[russian]{babel}
\usepackage[utf8]{inputenc}
\usepackage{amsmath}
\usepackage{amsfonts}
\usepackage{enumitem}
\usepackage{ntheorem}
\usepackage{tikz}
\usepackage{verbatim}

\sloppy

\usetikzlibrary{arrows,shapes}

\tikzstyle{vertex}=[circle,fill=black,minimum size=3pt,inner sep=0pt]
\tikzstyle{edge} = [draw,thick,-]

\newtheorem{definition}{Определение}
\newtheorem*{solution}{Решение}

\newenvironment{task}[2] {
	\noindent\fbox{\bf {#1} {#2}.}
}{
}

\newcommand{\mytitle}[2] {
  \begin{center}
      \bf {#1} {#2}.
  \end{center}
}

\let\origenumerate\enumerate
\let\origendenumerate\endenumerate
\renewenvironment{enumerate}{\origenumerate[topsep = 0pt, noitemsep]}{\origendenumerate}

\begin{document}
	\mytitle{Домашняя работа 5.}{Теория графов}
	\begin{task}{COMB}{64}
		(1 балл) Постройте пример такого графа $G$, что его графова последовательность
		равна $(n, n, n - 1, n - 1, \dots, 2, 2, 1, 1)$.
	\end{task}
	\begin{solution}
	\end{solution}

	\begin{task}{COMB}{65}
		(2 балла) Собственным числом графа $G$ называется собственное число матрицы $M_a$
		смежности этого графа ($\lambda$ --- это собственное число матрицы $M$, если существует такой вектор $v$, что $Mv = \lambda v$). 
		
		Доказать, что $k$-регулярный граф $G$ имеет собственное число $\lambda = k$.
	\end{task}
	\begin{solution}
	\end{solution}

	\begin{task}{COMB}{66}
		(1 балл) Доказать, что для произвольного турнира T справедливо равенство $\sum\limits_{v \in V(T)} (\mathrm{indeg(v)})^2 = \sum\limits_{v \in V(T)} (\mathrm{outdeg(v)})^2$.
	\end{task}
	\begin{solution}
	\end{solution}

	\begin{task}{COMB}{67}
		(2 балла) Орграф $D$ называется сбалансированным, если для любой вершины $x \in V(D)$
		выполняется неравенство $|\mathrm{outdeg}(x) - \mathrm{indeg}(x)| \le 1$.
		
		Доказать, что из любого неориентированного графа $G$ можно получить направленный сбалансированный орграф $D$
	\end{task}
	\begin{solution}
	\end{solution}

	\begin{task}{COMB}{68}
		(1 балл). Пусть $G$ есть граф, построенный на вершинах $1$, \dots, $15$, в котором вершины 
		$i$ и $j$ смежны тогда и только тогда, когда их наибольший общий делитель больше единицы.
		Подсчитать количество связных компонент такого графа, а также определить максимальную
		длину простого пути в графе $G$.
	\end{task}
	\begin{solution}
	\end{solution}

	\begin{task}{COMB}{69}
		(1 балл). Доказать, что в связном графе два максимальных простых пути имеют общую
		вершину.
	\end{task}
	\begin{solution}
	\end{solution}

\end{document}